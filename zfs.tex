\documentclass{book}

\usepackage{xcolor}
\usepackage{listings}

\lstset{ %
 basicstyle=\ttfamily,
 breaklines=true,
 escapechar=|,
}

\begin{document}

\begin{titlepage}
\title{ZFS Administration}
\author{Aaron Toponce}
\maketitle
\end{titlepage}

\chapter{Installing ZFS on Linux}
First, we'll start with installing ZFS as a kernel module, not FUSE, on
Debian GNU/Linux. The documents already exist for getting this going, I’m
just hoping to spread this to a larger audience, in case you are unaware
that it exists.

First, the Lawrence Livermore National Laboratory has been working on
porting the native Solaris ZFS source to the Linux kernel as a kernel
module. So long as the project remains under contract by the Department of
Defense in the United States, I’m confident there will be continuous
updates. You can track the progress of that porting at
http://zfsonlinux.org.

\begin{lstlisting}
|\$| su -
# wget http://archive.zfsonlinux.org/debian/pool/main/z/zfsonlinux/zfsonlinux_1%7Ewheezy_all.deb
# dpkg -i zfsonlinux_1~wheezy_all.deb
# apt-get update
# apt-get install debian-zfs
\end{lstlisting}

And that’s it!

If you’re running Ubuntu, which I know most of you are, you can install the
packages from the Launchpad PPA https://launchpad.net/\~zfs-native.

Now, make your zpool, and start playing:

\begin{lstlisting}
|\$| sudo zpool create test raidz sdd sde sdf sdg sdh sdi
\end{lstlisting}

It is stable enough to run ZFS on production data. It is copy-on-write,
supports compression, deduplication, file atomicity, off-disk caching,
encryption, and much more. At this point, unfortunately, I’m convinced that
ZFS as a Linux kernel module will become “stable” long before Btrfs will be
stable in the mainline kernel. Either way, it doesn’t matter to me. Both
are Free Software, and both provide the long needed features we’ve needed
with today’s storage needs. Competition is healthy, and I love having
choice. Right now, that choice might just be ZFS.

\end{document}
